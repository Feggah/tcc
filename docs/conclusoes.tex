\chapter{Conclusões}\label{chp:conclusoes}

Este trabalho apresentou uma abordagem automatizada de rastreamento de contatos. Foi proposta uma solução baseada em um aplicativo de dispositivos móveis, disponível para a maioria dos sistemas operacionais, permitindo a utilização do produto por grande parte da população.

Por meio de testes com diferentes dispositivos foi validado que a solução está funcional e possui a capacidade de notificar as pessoas possivelmente expostas à alguma doença infecciosa.

Com isso, o aplicativo conseguiu resolver o problema manual que foi relatado no Capítulo \ref{chp:Introducao}, transformando o alto número de pessoas para efetuar o rastreamento de contatos com cada infectado em uma solução digital automatizada. A única necessidade manual é por parte dos usuários, que devem fazer os \textit{check-ins} e declararem no aplicativo quando forem infectados, para que as rotinas de análise sejam ativadas.

\section{Trabalhos futuros}
As sugestões de trabalhos futuros baseiam-se na implementação de mais funcionalidades ao aplicativo, destacando-se:
\begin{itemize}
    \item A substituição da utilização de \textit{check-ins} por \textit{Bluetooth}. Essa solução faria com que o rastreamento se tornasse totalmente automático, porque remove a necessidade do usuário fazer os \textit{check-ins} de suas visitas, e pode ser cumprida através do desenvolvimento de uma biblioteca própria ou da comunidade com as funcionalidades necessárias do \textit{Bluetooth}.
    \item Adicionar suporte para que o usuário consiga deletar os dados salvos no aplicativo, isso significa dar suporte a deleção dos locais salvos e a opção de deleção da conta;
    \item Construção de um painel administrativo, onde haverão gráficos de calor que extraem informações do BD, para que autoridades públicas possam tomar medidas de prevenção em locais onde estão sendo transmitidas doenças infecciosas em uma taxa crítica. Esse painel deve ser acessado somente por pessoas específicas e seria uma extensão total do sistema, necessitando outros mecanismos de autenticação, telas, bancos de dados e funcionalidades;
\end{itemize}

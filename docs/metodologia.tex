\chapter{Metodologia}\label{chp:metodologia}

Para elaboração deste trabalho, uma série de etapas foram realizadas. Essas etapas formam a metodologia seguida durante todo o processo de pesquisa e desenvolvimento do sistema.

% Discussão do tema com o orientador
Foi feita uma reunião com o orientador em que foi sugerido o problema a ser explorado pelo trabalho e quais possíveis caminhos poderiam ser seguidos para encontrar uma solução.

% Pesquisas sobre o rastreamento de contatos, entender o que é feito e como resolver o problema atual
Após essa reunião, foi realizada uma pesquisa de exploração de soluções existentes em relação ao rastreamento de contatos, utilizando ferramentas de busca na \textit{Internet}. A partir dessa busca, foi identificado o problema explicitado no Capítulo \ref{chp:Introducao} e algumas hipóteses foram levantadas para serem possíveis soluções.

% Análise de aplicativos semelhantes nas loja de aplicativo Play Store
Antes de começar o levantamento e análise de requisitos, foi realizada uma busca no Instituto Nacional da Propriedade Industrial, buscando por patentes relacionadas ao tema deste projeto. Além disso, soluções atuais para o problema também foram analisadas, elas foram encontradas na loja oficial de aplicativos para celulares \textit{Android}. Os resultados dessas buscas estão explicitados no Capítulo \ref{chp:correlatos}.

% Entender cada possibilidade de rastreamento e escolher a que melhor se encaixa na solução
% Levantamento e análise de requisitos
O levantamento e análise de requisitos foi feito a partir da solução proposta e de funcionalidades analisadas em outras aplicações atuais disponíveis no mercado. Como consequência dos requisitos, foi escolhida a solução a ser explorada pelo trabalho. Todo levantamento e análise estão contidos na Seção \ref{sec:requisitos}.

% Pesquisas sobre arquitetura de software para entender qual a melhor arquitetura para ser utilizada no aplicativo
Foram feitas pesquisas referentes a diferentes arquiteturas de \textit{software}, procurando por uma que atenda aos casos de uso do projeto e ofereça boa qualidade de código. A arquitetura escolhida para o desenvolvimento está explicada na Seção \ref{sec:clean}.

% Pesquisas sobre diagramação UML para entender quais melhor se encaixam com cada tipo de modelagem a ser feita
Pesquisas em relação à  \Sigla{\textit{Unified Modeling Language}}{UML} foram realizadas, com o objetivo de entender e levantar quais diagramas deveriam ser modelados para melhor análise do sistema. Os diagramas utilizados e os motivos de escolha de cada um estão elencados na Seção \ref{sec:uml}.

% Modelagem da arquitetura
% Modelagem da estrutura dos dados do BD
% Modelagem da aplicação como um todo (software - fila - banco - cloud function)
Após isso, a modelagem dos diagramas foi realizada utilizando os diagramas de caso de uso, de sequência, classes, pacotes, atividade e componentes. A maioria desses diagramas se encontram na Seção \ref{sec:modelagem}. Além de diagramas UML, também foi feita a modelagem de dados do BD.
% https://www.ibm.com/docs/pt-br/rsas/7.5.0?topic=topologies-deployment-diagrams

% Pesquisas sobre opções de infraestrutura para sustentar a solução da aplicação
Pesquisas em relação à infraestrutura foram realizadas, como o sistema possui um \textit{backend} e um BD remoto. Essa pesquisa teve como objetivo entender quais componentes eram necessários, as suas responsabilidades e como interagem entre si. A explicação sobre cada um dos componentes está explicitada no Capítulo \ref{chp:infraestrutura}.

% Pesquisas sobre fluxos de trabalho com a ferramenta Git
Depois de toda análise de requisitos e diagramação do sistema, uma pesquisa sobre fluxos de trabalho utilizando a ferramenta \textit{Git} foi feita, com o objetivo de escolher o melhor método de trabalho com integração contínua, separando cada \textit{branch} de desenvolvimento com sua própria responsabilidade e aumentando ainda mais a qualidade do código desenvolvido. O funcionamento do fluxo de trabalho escolhido está explicitado na Seção \ref{sec:Git}.

% Criação de telas no Figma
Todas as telas do aplicativo foram construídas utilizando o programa \textit{Figma}. A Seção \ref{sec:uiux} contém exemplos de algumas interfaces prototipadas.

% Pesquisas sobre linguagens de programação para entender qual é a melhor para ser utilizada nesse caso
Antes de começar o desenvolvimento, pesquisas sobre quais linguagens de programação devem ser utilizadas no sistema foram realizadas, tendo em vista todos os requisitos levantados, a arquitetura e infraestrutura do sistema. Essas linguagens devem possuir suporte as decisões tomadas anteriormente. Na Seção \ref{sec:flutter} está explicado os motivos da decisão da linguagem a ser utilizada.

% Pesquisas sobre princípios de desenvolvimento de software, buscando melhoria na qualidade do código
Com a linguagem escolhida, a última pesquisa a ser feita antes da implementação foi sobre princípios de desenvolvimento de \textit{software}. Essa pesquisa foi importante para que o sistema possua alta qualidade, legibilidade e outras características detalhadas na Seção \ref{sec:solid}.

% Pesquisas sobre quais bibliotecas seriam utilizadas no app
% Codificação do app
Por fim, o desenvolvimento do sistema foi feito, juntamente com pesquisas de bibliotecas de auxílio para facilitar algumas operações específicas da aplicação. O desenvolvimento do sistema é composto pela codificação do aplicativo móvel, do \textit{backend} e de \textit{pipelines} responsáveis pela integração contínua. A Seção \ref{sec:implementacao} apresenta os pontos mais importantes do processo de desenvolvimento.

%%%%%%%%%%%%%%%%%%%%%%%%%%%%%%%%%%%%%%%%%%%%%%%%%%%%%%%%%%%%%%%%%%
% The following comments were written in Portuguese, because this 
% template applies only for School of Technology at University 
% of Campinas, Brazil.
%
% Este é um modelo Latex para monografias de Trabalhos de Conclusão 
% de Curso (TCC) na graduação, monografias de Mestrado e Teses de 
% doutorado da Faculdade de Tecnologia (FT) da Universidade 
% Estadual de Campinas (UNICAMP).
%
% Esse modelo e seu respectivo arquivo de classe de documento 
% foram adaptado do modelo de teses e dissertações do 
% Instituto de Computação da UNICAMP e estão de acordo com a 
% Norma CPG 001/2019.
%
% Autor: André Leon Sampaio Gradvohl, Dr.
% Email:     gradvohl@ft.unicamp.br
% Lattes CV: http://lattes.cnpq.br/9343261628675642
% ORCID:     0000-0002-6520-9740
% 
% Última versão: 14/junho/2020
%
% Adições/Alterações nesta última versão:
% - Adição de informações sobre a inclusão do arquivo com a 
%   ficha catalográfica ou página em branco.
% - Ajustes no idioma para a lista de abreviações e siglas.
% - Adequações na lista de símbolos.
% - Inclusão do pacote silence para suprimir os "warnings" 
%   referentes à inclusão do arquivo "brazilian-abnt". 
% - Adição da opção "noFig" para acelerar a compilação, sem 
%   a adição de figuras. Use apenas junto com a opção Draft.
% - Adição de comandos para referenciar múltiplos capítulos, 
%   seções, figuras, tabelas, equações.
% - Adição de citação a software na bibliografia.
%%%%%%%%%%%%%%%%%%%%%%%%%%%%%%%%%%%%%%%%%%%%%%%%%%%%%%%%%%%%%%%%%%
%
% Escolha: Portugues ou Ingles ou Espanhol.
% Para a versão final do texto, acrescente a palavra "Final".
%\documentclass[Portugues,Final]{tese-FT}
%\documentclass[Ingles,Final]{tese-FT}
%\documentclass[Espanhol,Final]{tese-FT}
%
% Para uma compilação mais rápida, utilize a opção "Draft", em
% letras maiúsculas como a seguir. 
\documentclass[Portugues,Draft]{tese-FT}
% Caso necessário, adicione também a opção "noFig". Essa opção 
% deixa de mostrar as figuras e, portanto, acelera mais ainda
% a compilação.
%\documentclass[Portugues,Draft,noFig]{tese-FT}


%Adicione seu arquivo com as referências bibliográficas
\addbibresource{bibliografia.bib}

%O pacote a seguir gera um dummy text. Elimine a linha quando
% for editar seu texto.
\usepackage{lipsum}

\begin{document}

% Escolha entre autor ou autora:
\autor{Gabriel Domingues Ferreira}
%\autora{Nome da Autora}

% Sempre deve haver um título em português:
\titulo{RASTREAMENTO DE EXPOSIÇÃO À DOENÇAS INFECCIOSAS POR APLICATIVO CELULAR}

% Se a língua for o inglês ou o espanhol defina:
%\title{The Dissertation or Thesis Title in English or Spanish for FT}

% Escolha entre orientador ou orientadora e inclua os títulos:
\orientador{Prof. Dr. Plínio Roberto Souza Vilela}
%\orientadora{Profa. Dra. Nome da Orientadora}

% Escolha entre coorientador ou coorientadora, se houver, 
% e inclua os títulos:
%\coorientador{Prof. Dr. Eng. Lic. Nome do Co-Orientador}
%\coorientadora{Prof. Dra. Eng. Lic. Nome da Co-Orientadora}

% Escolha entre uma das seis opções a seguir (comente as demais):
\bsi         % para Trabalho de Conclusão de Curso em BSI
%\tads        % para Trabalho de Conclusão de Curso em TADS
%\qualificacaoMestrado  % Para textos de qualificação de mestrado.
%\qualificacaoDoutorado % Para textos de qualificação de doutorado.
%\mestrado   % para Dissertação de Mestrado em Tecnologia
%\doutorado  % para Tese de Doutorado em Tecnologia

%Defina a área de concentração. Se for TCC, deixe comentado.
%\areaConcentracao{Sistemas de Informação e Comunicação}
%\areaConcentracao{Ambiente}
%\areaConcentracao{Ciência dos Materiais}

% Se houve cotutela, defina:
%\cotutela{Universidade Nova de Plutão}

% Defina a data da defesa no formato {Dia}{Mês}{Ano}
% Use apenas números! O template transformará em palavras,
% se necessário.
\datadadefesa{12}{6}{2020}

% Para a versão final defina:
% Repita o nome do Orientador(a) no primeiro avaliador
\avaliadorA{Prof. Dr. Plínio Roberto Souza Vilela}{FT/UNICAMP}
\avaliadorB{Profa. Dra. Segunda Avaliadora}{Instituição da segunda avaliadora}
\avaliadorC{Dr. Terceiro Avaliador}{Instituição do terceiro avaliador}
% \avaliadorD{Prof. Dr. Quarto Avaliador}{Instituição do quarto avaliador}
% \avaliadorE{Prof. Dr. Quinto Avaliador}{Instituição do quinto avaliador}
% \avaliadorF{Prof. Dr. Sexto Avaliador}{Instituição do sexto avaliador}
% \avaliadorG{Prof. Dr. Sétimo Avaliador}{Instituição do sétimo avaliador}
% \avaliadorH{Prof. Dr. Oitavo Avaliador}{Instituição do oitavo avaliador}

% Para incluir a ficha catalográfica em PDF na versão final, 
% copie o arquivo PDF para o projeto no Overleaf, descomente 
% e informe o nome do arquivo no comando a seguir.
% \fichacatalografica{SeuArquivo.pdf}
%
% Para deixar uma página em branco no lugar da ficha 
% catalográfica, descomente uma das três linhas a seguir:
\fichacatalografica{branco.pdf} % Português
%\fichacatalografica{white.pdf}  % Inglês
%\fichacatalografica{blanco.pdf} % Espanhol

% Este comando deve ficar aqui:
\paginasiniciais

% Se houver dedicatória, descomente a linha a seguir e 
% escreva a dedicatória em seguida.
% \prefacesection{Dedicatória}
% A dedicatória deve ocupar uma única página.
%

% Se houver epígrafe, descomente e edite as linhas a seguir:
% \begin{epigrafe}
% {\it
% Vita brevis,\\
% ars longa,\\
% occasio praeceps,\\
% experimentum periculosum,\\
% iudicium difficile.}
%
% \hfill (Hippocrates)
% \end{epigrafe}

% Adicione no arquivo "agradecimentos.tex" os seus agradecimentos
% Caso prefira omitir os agradecimentos, comente a linha a seguir.
% \input{agradecimentos}

% Sempre deve haver um resumo em português:
\begin{resumo}
O início do ano de 2020 foi marcado pela pandemia causada pelo coronavírus, e por conta disso, diversos países perceberam as ameaças que doenças infecciosas podem trazer à humanidade e voltaram suas atenções para combater sua transmissão. 
Uma das soluções para diminuir a alta taxa de transmissão da doença foi o rastreamento de contatos. Esse rastreamento acontece da seguinte forma: no momento que uma pessoa é diagnosticada como infectada, ela deve responder algumas perguntas sobre quem ela se encontrou nos últimos dias. As pessoas que forem rastreadas a partir dessas perguntas devem se isolar mesmo que não estejam apresentando sintomas, para impedir que o vírus continue se espalhando.

Essa é uma maneira pouco eficiente de rastreamento por dois principais motivos: o primeiro é que é totalmente manual, isso reflete na necessidade de muitos recursos humanos serem alocados para entrevistar os infectados, fazer ligações, envio de e-mails e mensagens para as pessoas que tiveram contato com ele. E quanto mais recursos envolvidos nesse processo, mais caro ele se torna. O segundo motivo é que o rastreamento depende inteiramente das lembranças do indivíduo, ou seja, se ele não se recordar de alguém que teve contato, o rastreamento acaba falhando em encontrar todas as pessoas que podem ter sido expostas.

Com base nesse contexto, neste trabalho foi proposto e desenvolvido um aplicativo para dispositivos móveis que armazena os lugares visitados pelos usuários e quando algum deles for infectado por alguma doença infecciosa, o aplicativo verificará automaticamente as pessoas que estiveram no mesmo local, no mesmo intervalo de tempo e as notificará, informando que podem ter sido expostas à doença e que devem se isolar para impedir novos casos de infecção.
\end{resumo}

% Sempre deve haver um abstract:
\begin{abstract}
The beginning of the year 2020 was marked by the pandemic caused by the coronavirus, and because of that, several countries realized the threats that infectious diseases can bring to humanity and turned their attention to combat its transmission.
One of the solutions to reduce the high rate of transmission of the disease was contact tracing. This tracing works as follows: the moment a person is diagnosed as infected, he must answer some questions about who he has met in the past few days. People who are traced for these questions should isolate themselves even if they are not showing symptoms, just to prevent the virus from spreading.

This is an inefficient way of tracking for two main reasons: the first is that it is completely manual, this reflects the need for many human resources to be allocated to interview the infecteds, make phone calls, send e-mails and messages to people who had contact with him. And the more resources involved in this process, the more expensive it becomes. The second reason is that the contact tracing depends entirely on the individual's memories, that is, if he does not remember someone who had contact, the tracing ends up failing to find all the people who may have been exposed.

Based on this context, in this project it was proposed and developed an application for mobile devices that stores the places visited by users and when one of them is infected by an infectious disease, the application will automatically check people who have been in the same place, in the same time and will notify them, stating that they may have been exposed to the disease and that they should isolate themselves to prevent new cases of infection.
\end{abstract}

% Se houver um resumo em espanhol, descomente as linhas a seguir:
%\begin{resumen}
% A mesma regra aplica-se.
%\end{resumen}

% A lista de figuras:
\listoffigures

% A lista de tabelas:
\listoftables

% A lista de símbolos é opcional. Não confunda a lista de símbolos 
% matemáticos com a lista de abreviaturas (que vem depois).
\input{listaSimbolos}

% A lista de abreviações e siglas vem a seguir.
% Dê uma olhada no pacote nomencl para ver os comandos para 
% adicionar abreviações e siglas no texto.
% Foram adicionados os comandos \Sigla{Sigla por extenso}{abrev} e 
% \SiglaHifen{Sigla por extenso}{abrev} para adicionar as siglas 
% diretamente no texto e criar a lisa de abreviaturas 
% automaticamente
\printnomenclature[3cm]

% O sumário vem aqui:
\tableofcontents

% E a linha a seguir deve ficar bem aqui. Não mude.
\fimdaspaginasiniciais

% O corpo da dissertação ou tese começa aqui:

\chapter{Introdução}\label{chp:Introducao}
% contexto do trabalho, motivações, objetivos e estrutura dos capítulos

A pandemia de COVID-19 mudou radicalmente as percepções do mundo em relação a propagação de doenças infecciosas como uma importante ameaça à humanidade. De acordo com o Instituto Ipsos \cite{Gebrekal2021}, o coronavírus se estabeleceu como a maior preocupação da população mundial por mais de um ano consecutivo.

Durante uma conferência realizada em Vancouver, \textcite{Gates2015} afirmou que especialistas alertavam que o mundo não estava preparado para o surgimento de um novo vírus com alta taxa de contágio. A facilidade que a humanidade possui em se locomover rapidamente para qualquer lugar do mundo através de aeroportos, ferrovias e estradas, a falta de um sistema de combate ao proliferamento de doenças infecciosas, métodos de prevenção e equipes de epidemiologistas altamente treinados contribuíram para o crescimento descontrolado dos casos de COVID-19 no ano de 2020.

Com o objetivo de diminuir a propagação do vírus, as mídias sociais foram bombardeadas com informações sobre métodos de prevenção, tais como lavar as mãos, evitar o compartilhamento de objetos pessoais, uso de máscaras, isolamento social e o rastreamento de contatos.

O rastreamento de contatos tem o objetivo de encontrar as pessoas que tiveram contato com as que foram infectadas pelo vírus, e instruí-las para não se encontrarem com família e amigos, ir aos mercados, bancos, ou seja, permanecerem isolamento social em suas casas, tudo isso com o objetivo de evitar que a corrente de transmissão do vírus continue e diminuir a sua taxa de contágio.

\section{Problema}\label{sec:problema}
Apesar do rastreamento reduzir a propagação do vírus avisando as pessoas expostas que elas podem estar infectadas, existia um grande problema: tudo é feito manualmente. Isso envolve um custo humano muito alto, são necessários muitos profissionais para que o rastreamento tenha algum resultado expressivo.

Além do alto custo para o governo manter a operação funcionando, ela depende integralmente da memória do paciente. Se o paciente entrevistado não lembrar de todas as pessoas que teve contato, ou se ele não souber os dados referentes às pessoas que ele esteve nas últimas semanas, isso afeta diretamente no resultado desse método de prevenção. Se muitas pessoas que foram expostas não forem identificadas, a corrente de transmissão continuará crescendo.

\section{Solução proposta}\label{sec:solucao}
Quando o número de pessoas envolvidas em uma solução é alto demais para arcar com os custos ou quando existe um alto risco humano envolvido, que nesse caso é o risco relacionado à memória do paciente, uma tecnologia de automação surge como solução, e nesse caso não é diferente.

A solução proposta é um aplicativo celular que será responsável tanto pela coleta das localizações visitadas quanto pela notificação de possíveis exposições aos usuários.

O aplicativo funcionará da seguinte maneira: o usuário deverá fazer \textit{check-in} nos locais que visitar e o aplicativo será responsável pelo armazenamento de cada local e o horário correspondente da visita. Caso algum usuário for infectado por alguma doença contagiosa, ele informará que está infectado e a automação fará uma busca no banco de dados. Caso outro usuário tenha visitado o mesmo local durante o mesmo intervalo de tempo, ele receberá uma notificação informando-o que pode ter sido exposto à uma doença infecciosa e sugerindo que evite sair de sua casa.

\section{Estrutura do texto}\label{sec:estrutura}

O projeto foi dividido em 7 capítulos e cada um deles possui uma responsabilidade específica.

O Capítulo \ref{chp:metodologia} explica quais foram os métodos utilizados para o desenvolvimento deste trabalho. O Capítulo \ref{chp:correlatos} contém a busca por patentes registradas e de aplicações correlatas ao tema deste projeto. No Capítulo \ref{chp:referencial} é feita a apresentação das tecnologias utilizadas e das pesquisas feitas sobre o tema. O Capítulo \ref{chp:desenvolvimento} contém o levantamento e análise de requisitos, a modelagem dos diagramas e da arquitetura utilizada no desenvolvimento e a explicação da implementação das funcionalidades. O Capítulo \ref{chp:infraestrutura} explica quais recursos de infraestrutura são necessários para que a aplicação funcione e qual a relação entre cada um deles. Por fim, o Capítulo \ref{chp:conclusoes} contém as conclusões do trabalho.



% Você pode usar o site http://www.tablesgenerator.com
% para gerar as tabelas em LaTeX.


% Note que podemos incluir automaticamente alguns termos na lista de abreviaturas e siglas no pré-texto. Veja exemplos a seguir no código fonte. 

% A sigla \Sigla{abêcê}{ABC} aparecerá automaticamente na lista de abreviaturas e siglas no pré-texto. Do mesmo modo, podemos usar o comando \SiglaHifen{xisipsilonzê}{XYZ} separado por hífen ao invés de aparecer entre parênteses.


% Aqui começa uma Seção.
% Use o comando \label para definir um rótulo, 
% caso seja necessário referenciar essa seção
% posteriormente.
%\section{Exemplo de seção}\label{sec:exemplo_secao} 

%Agora observe como se faz uma citação de artigo científico em periódico \cite{Gradvohl2014c}. De acordo com \textcite{Gradvohl2016}, essa é uma citação direta. Se for citar mais de um trabalho, faça da seguinte forma \cite{Caldana2017,Gradvohl2015}. As referências bibliográficas estão no arquivo \texttt{bibliografia.bib}. Outros exemplos de citações também se encontram nesse arquivo.

% Veja a seguir o comando para criar uma figura e o resultado, na \Figura{fig:xwing}. Note, no código fonte, que no comando \texttt{caption} podemos estabelecer uma \enquote{legenda curta} para aparecer na Lista de Figuras. A legenda curta é opcional.

%\begin{figure}[!htb]
%\centering
%As figuras estão na pasta figuras.
% Se seu texto estiver demorando muito para compilar, 
% use figuras no formato PDF ou PNG.
%\includegraphics[scale=0.2]{starwars21280.jpg}
%\caption[Legenda curta de figura]{Legenda mais extensa de figura.}
%\label{fig:xwing}
%\end{figure}

%\subsection{Exemplo de subseção}
%É importante evitar chegar a esse nível de subseção. Dois níveis são suficientes. Use essa opção em último caso, apenas.


%\subsection{Exemplo de adição de siglas}\label{subsec:siglas}
%Para adicionar uma sigla ou abreviatura na lista de siglas e abreviaturas, use o comando ``\texttt{\textbackslash{}Sigla\{nome por extenso\}\{abreviatura\}}'' ou ``\texttt{\textbackslash{}SiglaHifen\{nome por extenso\}\{abreviatura\}}'' para adicionar a sigla com hífen. 
%Por exemplo, respectivamente, \Sigla{Ácido Desoxirribonucleico}{DNA} ou \SiglaHifen{Ácido Ribonucleico}{RNA}. A lista de siglas é adicionada automaticamente.

%\section{Comandos opcionais para facilitar}
%Este modelo também criou alguns comandos adicionais não apenas para facilitar o trabalho de quem escreve, mas também para manter uma formatação mais consistente.

%Entre esses comandos estão o \texttt{\textbackslash{}ie} que inclui a abreviatura ``\ie'' no texto (equivalente ao ``isto é''). Usar esse comando vai garantir que a abreviatura não se separe entre linhas e que o espaço entre o `.' e a próxima letra seja fixo. O mesmo vale para os comandos \texttt{\textbackslash{}eg} que inclui a abreviatura ``\eg'' e \texttt{\textbackslash{}pex} que inclui a abreviatura ``\pex''.

%Também existem os comandos \texttt{\textbackslash{}Capitulo\{rótulo do capítulo\}}, \texttt{\textbackslash{}Equacao\{rótulo da equação\}}, \texttt{\textbackslash{}Figura\{rótulo do figura\}}, \texttt{\textbackslash{}Secao\{rótulo da seção\}} e \texttt{\textbackslash{}Tabela\{rótulo da tabela\}}. Esses comandos inserem referências para os respectivos elementos. Além disso, no próprio texto aparece a \textit{string} (``Capítulo'', ``Equação'', ``Figura'' etc) seguida da referência já com o link. Por exemplo, \Secao{sec:exemplo_secao}. Sugere-se a utilização desses comandos para referenciar os respectivos elementos ao invés do comando \texttt{\textbackslash{}ref\{rótulo\}}. Assim, o texto ficará mais uniforme. 

%É possível também usar esses comandos nas versões no plurall para conjuntos de referências. Por exemplo, para referenciar várias seções, você pode utilizar o comando \texttt{\textbackslash{}secoes\{rótulo\_1, rótulo\_2, rótulo\_3\}}.

%Por exemplo, suponha que queiramos referenciar as \secoes{sec:exemplostabelas,sec:exemplo_secao,subsec:siglas}.

\chapter{Metodologia}\label{chp:metodologia}
% quais foram os passos seguidos por você para realizar o trabalho



\chapter{Trabalhos correlatos}\label{chp:correlatos}

Este capítulo trata das pesquisas realizadas procurando soluções no mercado que se propõem a resolver o problema apresentado pelo projeto e também consultas por algumas palavras-chave no BD do \Sigla{Instituto Nacional da Propriedade Industrial}{INPI}.

\section{Busca por patentes de software registradas}\label{sec:inpi}
A consulta no INPI foi feita utilizando a categoria de Programas de Computador, buscando os que tinham como parte de seu título as palavras-chave rastreamento de doenças, exposição a doenças, aplicativo de rastreamento ou rastreamento de contatos.

Levando em consideração a ordem de busca apresentada, apenas a terceira encontrou algum resultado, totalizando três respostas. Uma delas é um aplicativo de celular e as outras duas são aplicativos de análise de mercado da plataforma \textit{NinjaTrader8}.

Analisando esse aplicativo móvel, ele foi registrado por \textcite{moope} no dia 27 de outubro de 2020 e se chama “MOOPE SISTEMA DE RASTREAMENTO E GESTÃO DE FROTAS - Módulo Aplicativo ios/Android, Módulo Frotas, Modulo Central Comandos, Modulo Central Notificações, Modulo MeusMoopes”, que tem como campo de aplicação a gestão de frotas. Portanto, o escopo deste aplicativo difere do proposto por este trabalho.

\section{Busca nas lojas de aplicativos}\label{sec:mercado}
No começo da pandemia do coronavírus, as empresas Google e Apple, detentoras dos dois principais sistemas operacionais para celulares - o \textit{iOS} e \textit{Android} - divulgaram o desenvolvimento de uma \Sigla{\textit{Application Programming Interface}}{API} para monitorar a exposição das pessoas ao coronavírus \cite{GoogleApple2020}. Esse monitoramento é feito a partir da captação do sinal entre pessoas próximas utilizando a tecnologia \textit{Bluetooth}. Essa API identifica se pessoas permaneceram próximas durante alguns minutos e se o sinal entre elas for suficientemente forte, cada dispositivo salva o identificador do outro que tiveram contato, assim seria possível fazer o rastreamento de contatos.

A API só pode ser consumida pelo governo dos países que se inscrevem no programa, ou seja, apenas os aplicativos desenvolvidos pelas autoridades públicas podem utilizar a tecnologia criada pelas duas empresas. Além disso, as empresas divulgaram que o recurso será desativado após a pandemia de coronavírus.

O único aplicativo que funciona no Brasil que possui a finalidade de rastreamento de contatos e que utiliza a API citada anteriomente foi desenvolvido pelo DATASUS e se chama "Coronavirus - SUS"  \cite{CoronavirusSUS}.

As principais funcionalidades deste aplicativo são que os usuários podem consultar as notícias oficiais compartilhadas pelo Ministério da Saúde, conferir as principais notícias falsas em circulação, informações sobre a doença, o compartilhamento de teste positivo e o rastreamento de contatos.

Além deste aplicativo, foram encontrados alguns desenvolvidos por outros países: o \textit{CDC}, \textit{Care19}, \textit{Crush Covid RI} e \textit{FEMA}. Nenhum desses aplicativos funcionam corretamente em território brasileiro, a única funcionalidade que pode ser aproveitada são as informações sobre a doença, escritas em inglês.

Com isso, o aplicativo que possui o objetivo de rastreamento de contatos e que funciona corretamente no Brasil é o Coronavírus - SUS.

As principais diferenças entre o aplicativo Coronavírus - SUS e o desenvolvido neste trabalho são: 
\begin{itemize}
    \item O aplicativo deste trabalho pode ser utilizado para qualquer tipo de doença infecciosa, enquanto o outro possui o foco no coronavírus;
    \item A API que o aplicativo Coronavírus - SUS utiliza será desativada após a pandemia, já a solução deste trabalho não possui nenhum componente temporário, podendo ser utilizado mesmo após a pandemia;
    \item O método de rastreamento do aplicativo deste trabalho utiliza o histórico de locais visitados e busca encontros entre os usuários, já o do SUS utiliza o sinal \textit{Bluetooth} para detectar o contato entre os usuários;
\end{itemize}


\chapter{Referencial teórico}\label{chp:referencial}
% Para realização deste trabalho, é necessário o desenvolvimento de um aplicativo móvel para que o rastreamento de doenças infecciosas ocorra de forma automática. Assim, na seção X ...
Para sustentar o desenvolvimento deste trabalho, padrões de desenvolvimento, de modelagem, linguagens de programação e outros assuntos foram pesquisados. Assim, a Seção \ref{sec:infecciosas} trata das pesquisas realizadas em relação à doenças infecciosas, procurando entender quais são os tipos de doenças que o aplicativo pode ajudar no rastreamento. Na Seção \ref{sec:uml} são elencados os tipos de diagramas UML foram utilizados e os motivo dessa decisão. Na Seção \ref{sec:clean} é explicado qual a arquitetura de \textit{software} utilizada na codificação do aplicativo, como ela funciona e quais são os benefícios de sua utilização. A Seção \ref{sec:repositorypattern} possui a explicação do Padrão de Repositório. A Seção \ref{sec:flutter} possui o propósito de explicar qual foi o \textit{framework} de desenvolvimento escolhido e o motivo de sua escolha. Na Seção \ref{sec:solid} é explicado cada um dos princípios utilizados no processo de desenvolvimento da aplicação. A Seção \ref{sec:eventos} trata da arquitetura utilizada no processo de \textit{design} do sistema, explicando seu funcionamento e benefícios. Na Seção \ref{sec:bd} são elencadas as principais características do tipo de BD escolhido para o sistema. A Seção \ref{sec:Git} explica sobre o fluxo de trabalho baseado na ferramenta \textit{Git} que foi utilizado no desenvolvimento do projeto. Na Seção \ref{sec:pipelines} os benefícios de utilização de \textit{pipelines} de integração e entrega contínuas são elencados, juntamente com os objetivos de utilização dessa ferramenta. A Seção \ref{sec:googlefirebase} explica o motivo de utilização de serviços na nuvem, quais foram necessários e os principais motivos dessa escolha.

\section{Doenças infecciosas}\label{sec:infecciosas}
% Aplicativo suporta doenças infecciosas transmitidas pelo ar
O rastreamento de contatos que o aplicativo automatizará não serve como método de prevenção e controle de qualquer doença transmissível. Por isso, pesquisas relacionadas à doenças infecciosas foram realizadas para entender quais são os tipos que o aplicativo ajudará na prevenção.

Segundo \textcite{Duncan2013}, a transmissão de doenças contagiosas podem ocorrer de forma direta ou indireta. A transmissão direta é a transferência direta e imediata de agentes infecciosos a uma porta de entrada receptiva no hospedeiro, que ocorre no contato direto com pele e mucosas. 

A transmissão indireta ocorre por meio dos três mecanismos primários: veículos, vetores e ar. Veículo é qualquer objeto que sirva como meio pelo qual o agente infeccioso se transporta a um hospedeiro, alguns exemplos são os brinquedos, utensílios de cozinha e instrumentos cirúrgicos. Vetores são artrópodes, que são divididos entre vetores mecânicos e biológicos. A transmissão aérea ocorre quando partículas que contem os agentes infecciosos ficam suspensas no ar por longos períodos de tempo.

As doenças contagiosas que o rastreamento de contatos efetuado pelo aplicativo celular suporta são relacionadas as de transmissibilidade indireta, principalmente as que ocorrem através do ar. Com isso, alguns exemplos de doenças contagiosas que o aplicativo pode suportar são o coronavírus, a gripe, sarampo e a catapora \cite{Descomplica}.

\section{\textit{Unified Modeling Language}}\label{sec:uml}
% UML
Como parte do processo de análise do sistema, as diagramações foram feitas seguindo os padrões UML, que é uma linguagem de modelagem utilizada para especificar, visualizar e documentar modelos de sistemas de \textit{software}, incluindo sua estrutura e \textit{design} \cite{uml}.

Existem diversos tipos de diagramas UML e eles são dividos em duas categorias: os estruturais e o comportamentais. A \Figura{fig:umloverview} representa os tipos de diagrama de cada uma das categorias.

\begin{figure}[!htb]
    \centering
    \includegraphics[scale=0.55]{uml-overview.png}
    \caption{Tipos de diagramas UML}
    \label{fig:umloverview}
\end{figure}

A partir dos tipos de diagramas disponíveis, foram utilizados apenas seis deles. Esses seis diagramas foram escolhidos a partir do objetivo que cada modelagem deveria atender. 

O primeiro deles tem a necessidade de modelar as funcionalidades necessárias para os atores que estão envolvidos com o sistema, representando os seus casos de uso. Por isso, esta diagramação foi feita utilizando o diagrama de casos de uso.

Para cada funcionalidade, foi necessário um diagrama que representasse a sequência de trocas de mensagens entre os objetos do sistema, e o diagrama UML utilizado que possui esse propósito é o diagrama de sequência.

Com o propósito de clarificar o que uma funcionalidade deve fazer, trazendo melhor documentação e explicação sobre a mesma, foi utilizado o diagrama de atividades. Esse diagrama é, essencialmente, um gráfico de fluxo que mostra o fluxo de controle entre uma atividade para outra. Neste trabalho, isso envolve a modelagem das etapas sequenciais de uma funcionalidade.

Como o sistema foi desenvolvido utilizando programação orientada a objetos, um diagrama para representar as classes, atributos, métodos e o relacionamento entre elas foi necessário. O diagrama que atende essa necessidade e foi utilizado no trabalho é o diagrama de classes.

Para representar a arquitetura de software do aplicativo móvel desenvolvido foi utilizado o diagrama de pacotes. Ele descreve os pacotes do sistema mostrando as dependências entre eles e o agrupamento de suas classes.

Por último, um diagrama que abrangesse o sistema como um todo, que represente o aplicativo cliente, o BD e outros componentes do sistema foi necessário. O diagrama de componentes tem exatamente esse propósito. Este diagrama mostra o relacionamento entre diferentes componentes de \textit{software} do sistema.

O principal objetivo e benefício da aplicação dessas modelagens é trazer eficiência no processo de desenvolvimento do sistema. Como toda análise foi feita anteriormente, essa eficiência é consequência da facilidade de visualização do sistema, documentação das decisões tomadas e ajuda a entender um sistema complexo dividindo-o em pequenas partes bem estruturadas.

\section{Arquitetura limpa}\label{sec:clean}

A arquitetura limpa, ou comumente conhecida como \textit{Clean Architecture}, foi criada por \textcite{cleanarchitecture} em seu livro \textit{Clean Architecture: A Craftsman’s Guide to Software Structure} e ela fornece uma metodologia de desenvolvimento que facilita o desenvolvimento de código, permite melhor atualização, manutenção e menos dependências entre os componentes do \textit{software}.

Um dos objetivos da arquitetura limpa é o princípio de \textit{design} de código conhecido como \Sigla{\textit{Separation of Concerns}}{SoC}. Esse objetivo é atingido a partir da separação do \textit{software} em camadas com diferentes responsabilidades.

Algumas das vantagens dessa separação por camadas são:

\begin{itemize}
    \item Testabilidade: As regras de negócio desenvolvidas no sistema poder ser testadas isoladamente, ou seja, sem  \Sigla{\textit{User Interfaces}}{UI}, bancos de dados ou agentes externos;
    \item Independente de UI: As interfaces podem ser modificadas frequentemente sem haver a necessidade de modificar o resto do \textit{software};
    \item Independente de BD: O sistema não está preso a nenhum tipo de BD específico e pode ser trocado sem afetar as regras de negócio;
    \item Independente de agentes externos: As regras de negócio não conhecem nada externo ao sistema;
\end{itemize}

\begin{figure}[!htb]
    \centering
    \includegraphics[scale=0.55]{CleanArchitecture.jpg}
    \caption[Arquitetura limpa]{Arquitetura limpa. Fonte: \cite{clean}}
    \label{fig:clean1}
\end{figure}

Cada círculo da \Figura{fig:clean1} representa diferentes áreas do sistema. Essas áreas só podem depender de outras mais internas, ou seja, a dependência só pode ocorrer "para dentro" do diagrama. A partir disso, nenhuma área conhece outras que sejam externas à ela. 

Essa regra de dependência, representada pelas setas da \Figura{fig:clean1},   garante que o sistema seja totalmente testável e poupa o desenvolvedor de problemas futuros com manutenções. Como o sistema é independente de camadas externas, como por exemplo \textit{frameworks} e bancos de dados, no momento que for necessário efetuar a mudança destes, o processo será mais simples e garantirá a integridade das regras de negócio.

Segundo \textcite{clean} e levando em consideração a legenda da \Figura{fig:clean1}, cada camada possui as seguintes definições:

\begin{itemize}
    \item \textit{Enterprise Business Rules}: são os objetos de domínio da aplicação que encapsulam as regras gerais e de alto nível. Eles aplicam lógicas gerais para toda a entidade;
    \item \textit{Application Business Rules}: são as ações do negócio, ou seja, contém as regras de negócio dos casos de uso do sistema. Esses casos de uso orquestram o fluxo de dados que são enviados e recebidos das entidades, direcionando-as para atingir um objetivo específico do caso de uso;
    \item \textit{Interface Adapters}: Essa camada é responsável por transformar as estruturas de dados dos casos de uso para o formato adequado aos agentes externos, como bancos de dados ou \textit{frameworks}, ou o contrário, convertendo os dados vindos dos agentes externos para a estrutura esperada pelos casos de uso;
    \item \textit{Frameworks \& Drivers}: Essa camada é composta por \textit{frameworks} ou ferramentas, como o BD. Nela é feita a ligação entre os agentes externos com a próxima camada mais interna ao círculo;
\end{itemize}

A arquitetura limpa não implica regras de apenas utilizar essas quatro camadas citadas anteriormente, isso significa que existem casos em que mais camadas podem ser utilizadas. Entretanto, a regra de dependência sempre deve ser aplicada mesmo nas novas camadas, e quanto mais interna ela for, mais abstrata ela deve se tornar.

A regra de dependência é clara, mas existem situações em que uma camada interna precisa chamar uma externa à ela. Por exemplo, o canto inferior direito da \Figura{fig:clean1} mostra os \textit{controllers} e \textit{presenters} se comunicando através da camada de caso de uso. Analisando o fluxo, ele começa no \textit{controller}, passa pelos casos de uso e termina no \textit{presenter}.

Essa violação da regra de dependência, onde o caso de uso invoca o \textit{presenter}, é resolvida através do princípio de inversão de dependências. Para isso, abstrações são utilizadas para que as dependências do código-fonte se oponham ao fluxo apenas nas fronteiras das camadas.

Considere o exemplo da necessidade do caso de uso chamar o \textit{presenter}. O caso de uso não deve chamar o \textit{presenter} diretamente, porque nesse caso ele violaria a regra de dependência. Então o caso de uso chama uma interface em sua camada, e as implementações dos métodos expostos pela interface são feitas pela camada \textit{presenter}.

Com isso, através do polimorfismo são criadas dependências no código que se opõem ao fluxo de controle para que o código esteja em conformidade com a regra de dependência, independente da direção que o fluxo de controle está indo.

Construindo a aplicação a partir dessas regras e princípios fazem com que o sistema seja testável, de fácil manutenção, já que a troca de \textit{frameworks} e bancos de dados podem ser realizadas sem impactar as camadas internas, criações de novas funcionalidades e refatorações podem ser feitas com mais facilidade e segurança, porque o código estará coberto de testes e a separação por camadas diminui o acoplamento, que minimiza o impacto de cada modificação.

\section{Padrão de Repositório}\label{sec:repositorypattern}

O padrão de repositório é um \textit{framework} conceitual responsável por encapsular um conjunto de objetos persistidos em um armazenamento de dados e as operações realizadas sobre eles, provendo uma visão orientada à objetos da camada de persistência da aplicação \cite{patterns}.

Segundo \textcite{ddd}, alguns dos benefícios da utilização desse padrão são:

\begin{itemize}
    \item Interface simples para obter e gerenciar o ciclo de vida de objetos persistidos;
    \item Desacopla a camada de domínio da aplicação da camada de persistência de dados;
    \item Permite fácil substituição da implementação, facilitanto o desenvolvimento de testes;
\end{itemize}

O repositório será utilizado para atender a regra de dependência da arquitetura limpa, fazendo com que os casos de uso não dependam da implementação das consultas feitas aos agentes externos, que estão presentes em camadas mais exteriores.

No processo de modelagem da arquitetura da aplicação, presente na Seção \ref{sec:modelagem}, é explicitado com detalhes a maneira que o padrão será utilizado no sistema.

\section{\textit{Frameworks} híbridos}\label{sec:flutter}
% Flutter 
Um dos principais requisitos do projeto é que o aplicativo deve funcionar em diversos tipos de sistemas operacionais de dispositivos móveis. Os dois principais do mercado são o \textit{Android} e o \textit{iOS}, que representam a grande maioria dos dispositivos.

Com isso, as linguagens de programação nativas de cada sistema foram excluídas, e os dois principais \textit{frameworks} híbridos utilizados no mercado são o \textit{Flutter} e o \textit{React}.

A decisão pela utilização do \textit{Flutter} aconteceu por alguns fatores. O principal deles foi o fato deste \textit{framework} ter sido lecionado durante uma disciplina cursada na graduação, outro fator foi em relação a existência de uma grande comunidade ativa e receptiva, que como consequência disso já produziu muitas documentações e tutoriais envolvendo a solução, e por último, a curva de aprendizado, que por conta da experiência prévia e da grande quantidade de materiais disponíveis na internet, se torna menor quando comparada ao \textit{React}.

\section{Princípios SOLID}\label{sec:solid}
% Principios SOLID
Os princípios SOLID serão utilizados no desenvolvimento do aplicativo por conta dos grandes benefícios que sua aplicação traz ao desenvolvimento de software e em relação a qualidade do código resultante dessa aplicação. SOLID é um acrônimo mnemônico dos 5 princípios de programação orientada a objetos introduzidos por \textcite{unclebob} em seu artigo \textit{Design Principles and Design Patterns}, onde cada letra representa um desses princípios.

Em seu artigo, Robert afirma que o software está em constante mudança e evolução, e conforme essa mudança acontece, a complexidade aumenta cada vez mais. Por conta disso, sem bons princípios de design de código, o software acaba se tornando de difícil manutenibilidade, testabilidade, legibilidade e extensibilidade. Os princípios SOLID foram desenvolvidos para combater esses problemas.

O objetivo geral desses princípios é reduzir as dependências para que diferentes áreas possam evoluir independentemente sem impactar umas às outras. Além disso, também possuem o propósito de construir designs fáceis de serem entendidos, mantidos, estendidos, escalados, reutilizados e testados. Por fim, a adoção dessas práticas evitam que problemas comuns no processo de desenvolvimento sejam enfrentados e possibilita a construção de software ágil, adaptável e eficiente.

Cada uma das letras do acrônimo representa um princípio, e são utilizados pela grande maioria dos sistemas desenvolvidos nos dias de hoje.

\subsection{Princípio da responsabilidade única}
A letra “S” do acrônimo representa o \textit{Single Responsibility Principle}, esse princípio tem relação direta com a coesão. Isso quer dizer que uma classe só terá uma única responsabilidade bem definida.

Em seu artigo, Robert descreve esse princípio dizendo que uma classe deve ter apenas um motivo para ser modificada; o que resulta na alta coesão citada no parágrafo anterior.

\subsection{Princípio aberto/fechado}
A letra “O” representa o \textit{Open/Closed Principle}, esse princípio foi descrito por Robert da seguinte forma: 

\begin{quotation}
"Devemos escrever nossos módulos de forma que possam ser estendidos, sem exigir que sejam modificados. Em outras palavras, queremos ser capazes de mudar o que os módulos fazem, sem alterar o código-fonte dos módulos."
\end{quotation}

Ou seja, Robert está se referindo ao conceito abstrato da extensão. Usar heranças ou interfaces que permitem polimorfismo é uma das maneiras mais comuns de cumprir esse princípio.

\subsection{Princípio substituição de Liskov}
A letra “L” do acrônimo significa \textit{Liskov Substitution Principle}, esse princípio prega que uma classe derivada deve ser substituível por sua classe base; essa frase é a simplificação da definição científica que foi apresentada por \textcite{liskov} no artigo \textit{Behavioral Subtyping Using Invariants and Constraints}.

A definição traduzida apresentada no artigo científico é:

\begin{quotation}
“Se $\theta(x)$ é uma propriedade provável dos objetos $x$ do tipo $T$. Então $\theta(y)$ deve ser verdadeiro para objetos $y$ do tipo $S$ onde $S$ é um subtipo de $T$.”
\end{quotation}

\subsection{Princípio segregação de interfaces}
A letra "I" representa o princípio \textit{Interface Segregation Principle}, em que Robert diz que muitas interfaces específicas para o cliente são melhores do que uma grande interface de propósito genérico.

Para os desenvolvedores, isso significa que novos métodos ou funcionalidades não devem ser criados a partir de uma interface existente, ao invés disso, é recomendado que se crie uma nova interface e as classes poderão implementar múltiplas interfaces conforme for necessário.

\subsection{Princípio da inversão de dependências}\label{sec:D}
A última letra do acrônimo, a letra "D", representa o princípio \textit{Dependency Inversion Principle}, ele oferece uma maneira de desacoplar módulos do \textit{software}.

Robert explica esse princípio dizendo que módulos de alto nível não devem depender dos de baixo nível, ambos devem depender de abstrações. Além disso, as abstrações não devem depender dos detalhes, os detalhes que devem depender das abstrações.

\section{Arquitetura orientada a eventos}\label{sec:eventos}
% Arquitetura orientada a eventos
Segundo a \textcite{redhat}, a arquitetura orientada a eventos é um modelo de \textit{design} de sistemas que não depende de linguagens de programação ou \textit{frameworks}, porque nele é abordado como modelar seu sistema de maneira agnóstica à linguagens.

Aplicando essa arquitetura, o acoplamento entre serviços no sistema é mínimo, ou seja, os diferentes componentes do sistema não terão altas dependências diretas entre si. Isso acontece porque os produtores dos eventos não conhecem os consumidores e o evento em si não conhece as consequências de sua ocorrência. Portanto, o principal benefício trazido pela arquitetura é que o sistema se torna flexível, se adapta a mudanças e toma decisões em tempo real.

A definição de um evento é uma mudança de estado no \textit{software} ou \textit{hardware}. Quando essa mudança acontece, uma mensagem é enviada por uma parte do sistema para avisar outra parte que alguma mudança ocorreu.

Com base nas definições apresentadas, a arquitetura orientada a eventos é composta por produtores e consumidores de eventos. O papel do produtor é detectar eventos e os representar como uma mensagem, que é enviada por meio de canais, que serão posteriormente processadas de maneira assíncrona pelo consumidor. O consumidor detectará novas mensagens publicadas no canal de comunicação e processará esse evento a partir das informações contidas nele. 

Existem diferentes tipos de arquitetura orientada a eventos, o modelo utilizado neste trabalho será o \textit{pub/sub}. \textit{Pub/sub} é o nome utilizado para representar o modelo \textit{Publish/Subscribe}, trata-se de uma infraestrutura de mensageria baseada em subscrições em um fluxo de eventos, ou seja, após ocorrer um evento, ele será publicado em um sistema de mensagens assíncronas, em que haverá um consumidor que receberá a publicação.

A maneira que o sistema desenvolvido neste trabalho utiliza o modelo \textit{pub/sub} é explicado no Capítulo \ref{chp:desenvolvimento}.

\section{Banco de dados orientado a documentos}\label{sec:bd}
% Banco de dados orientado a documentos
O banco de dados orientado a documentos é um tipo de BD não-relacional que armazena e consulta dados em formato \Sigla{JavaScript Object Notation}{JSON}, cuja principal característica é a organização de dados livres de esquemas. Por conta disso, esse tipo de banco facilita as consultas feitas pelos desenvolvedores, que usam o mesmo formato de modelo que usam no código do aplicativo.

Outras características desse tipo de BD \textit{NoSQL} é a sua natureza semiestruturada, flexível e hierárquica dos documentos, que permite que o banco evolua conforme as necessidades do aplicativo.

\section{Fluxo de trabalho com \textit{Git}}\label{sec:Git}
% Fluxo de trabalho com Git
Este projeto foi desenvolvido utilizando um sistema de controle de versão chamado \textit{Git}. Ele foi escolhido por ser a ferramenta mais utilizada pela comunidade de desenvolvimento de \textit{software}, gratuita e de código aberto. A utilização de um sistema de versionamento foi por conta dos benefícios citados abaixo.

\begin{itemize}
    \item Registra todo o histórico de alterações dos arquivos. Isso significa que serão salvas todas as alterações de conteúdo, deleção ou adição dos arquivos, o autor, data e mensagens escritas em cada alteração.
    \item Facilidade na reversão de alterações, ou seja, caso uma nova funcionalidade não se comporte da maneira esperada e cause falhas no aplicativo, a reversão para alguma versão anterior é feita a partir do histórico registrado pelo versionamento.
    \item Suporte a ramificações e mesclas, que possibilitam a criação de diferentes linhas de desenvolvimento independente uma das outras, e posteriormente da mesclagem dessas linhas em uma ramificação principal, que conterá todo código testado e validado do projeto.
\end{itemize}

Apesar do \textit{Git} ser uma ferramenta robusta para o desenvolvimento, existem alguns tipos de fluxos de trabalhos que são adotados para desenvolver um projeto. Os fluxos de trabalho procuram padronizar a maneira que os desenvolvedores colaboram dentro do projeto, a nomeação de ramificações, a responsabilidade que cada tipo possui e a maneira que novas funcionalidades são integradas ao código principal.

Por conta de sua robustez, padronização e organização elevada, o fluxo de trabalho utilizado neste projeto foi o \textit{GitFlow}, criado por \textcite{gitflow}. Ele é ideal para gerenciar projetos grandes, com ciclo de lançamento agendado e atribui responsabilidades específicas para cada tipo de ramificação.

Existem 5 tipos de ramificações nesse fluxo de trabalho, sendo elas a \textit{main}, \textit{develop}, \textit{hotfix}, \textit{release} e \textit{feature}.

A \textit{main} e a \textit{develop} armazenam todo o histórico de alterações do desenvolvimento de um produto. Cada modificação na ramificação \textit{main} representa uma nova versão lançada para os usuários, e a ramificação \textit{develop} contém as iterações das funcionalidades que são desenvolvidas. Portanto, uma modificação na ramificação \textit{main} representa um conjunto de funcionalidades que foram iteradas na ramificação \textit{develop}.

A \Figura{fig:gitflow1} representa a evolução independente das duas ramificações, para melhor visualização desse diagrama e dos próximos, cada ramificação é representada por uma cor e a primeira letra de seu nome. 

\begin{figure}[!htb]
    \centering
    \includegraphics[scale=0.65]{gitflow-master-develop.png}
    \caption{Representação das ramificações \textit{main} e \textit{develop}}
    \label{fig:gitflow1}
\end{figure}

Para o desenvolvimento de novas funcionalidades é utilizada a ramificação \textit{feature}. Ela deve ser derivada da \textit{develop} e mesclada logo após a finalização do desenvolvimento da funcionalidade, como mostra a \Figura{fig:gitflow2}.

\begin{figure}[!htb]
    \centering
    \includegraphics[scale=0.65]{gitflow-feature.png}
    \caption{Representação das ramificações \textit{main}, \textit{develop} e \textit{feature}}
    \label{fig:gitflow2}
\end{figure}

Depois de algumas iterações de funcionalidades na \textit{develop}, é criado uma preparação para um lançamento de uma nova versão do produto. Essa preparação é feita pela ramificação \textit{release}. Nela, nenhuma funcionalidade é desenvolvida, apenas pequenos ajustes em possíveis falhas ou documentações, como mostra a \Figura{fig:gitflow3}.

É importante ressaltar que a \textit{release} é derivada da \textit{develop} e deve ser mesclada com a \textit{main} e com a  \textit{develop} novamente, já que algumas atualizações no código podem ter sido feitas na ramificação \textit{release}. Quando a mesclagem para a \textit{main} for feita, uma nova versão do produto é lançada.

\begin{figure}[!htb]
    \centering
    \includegraphics[scale=0.65]{gitflow-release.png}
    \caption{Representação das ramificações \textit{main}, \textit{develop}, \textit{feature} e \textit{release}}
    \label{fig:gitflow3}
\end{figure}

A última ramificação que esse fluxo de trabalho possui é a \textit{hotfix}. Ela é responsável por fazer pequenas alterações diretamente na main, ou seja, eventuais falhas que foram entregues aos usuários são corrigidas nesta ramificação.

A  \Figura{fig:gitflow4} mostra a \textit{hotfix} atualizando alguma falha que estava na \textit{main} e como todas as 5 ramificações estão presentes, consequentemente representa um ciclo completo utilizando o \textit{GitFlow}.

\begin{figure}[!htb]
    \centering
    \includegraphics[scale=0.65]{gitflow-final.png}
    \caption{Representação de todas ramificações do \textit{GitFlow}}
    \label{fig:gitflow4}
\end{figure}


\section{\textit{Pipelines} de integração e entrega contínua}\label{sec:pipelines}
% Pipelines de CI/CD
Como já citado anteriormente, a qualidade do software é uma preocupação importante no processo de desenvolvimento deste projeto. Aliado a isso, a criação de pipelines trás benefícios ao sistema, alguns deles são:
\begin{itemize}
    \item Maior taxa de entrega: em conjunto com o fluxo de trabalho adotado no desenvolvimento, a frequência com que novas funcionalidades ou correções de falhas são lançadas é maior. Como o processo de entrega é automatizado, o desenvolvedor foca no desenvolvimento de novas entregas, e não em passos repetitivos para efetuá-las;
    \item Confiabilidade de testes: os testes são efetuados de forma automática, e caso algum teste falhe, a entrega é impossibilitada de ser feita, garantindo que apenas pedaços de código validados possam ser entregues aos clientes;
    \item Redução de custos: menos tempo gasto em processos manuais e repetitivos, e mais tempo desenvolvendo novas funcionalidades resulta em redução de custos, já que o desenvolvedor se preocupa somente com atividades que agregam valor;
\end{itemize}

Os \textit{pipelines} de \Sigla{\textit{Continuous Integration and Continuous Delivery}}{CI/CD} são um conjunto de etapas que integram com o código principal, efetuando principalmente testes nesse novo código, e o entregam, que seriam novas versões liberadas aos clientes.

Existem diversas ferramentas de CI/CD, e a ferramenta escolhida foi o \textit{GitHub Actions}, porque possui integração perfeita com a central de repositórios \textit{git} que será utilizada, o \textit{GitHub}. Além disso, a ferramenta é gratuita, suporta as principais funcionalidades que um \textit{pipeline} precisa e possui boa documentação.

\section{\textit{Serviços computacionais na nuvem}}\label{sec:googlefirebase}
% Firebase
Durante o processo de desenvolvimento, surgiu a necessidade do sistema possuir serviços remotos na nuvem, tais como BD e um servidor \textit{backend}, onde rotinas pudessem rodar independentemente do dispositivo móvel do usuário.

Essa necessidade trouxe o requisito de se utilizar serviços computacionais na nuvem. Levando em consideração as tecnologias e os requisitos do aplicativo, o provedor utilizado foi o \textit{Firebase}.

Esse provedor de nuvem é especializado em acelerar o processo de desenvolvimento de sistemas \textit{mobile}, fornecendo ferramentas de fácil configuração, auto gerenciadas e de alta qualidade.

Os serviços que foram utilizados do provedor foi o BD orientado a documentos, o servidor \textit{backend} para executar as rotinas de rastreamento de contatos e um sistema de mensageria responsável por enviar ao \textit{backend} os eventos de escrita ocorridos em coleções específicas no BD.

Além da alta disponibilidade e confiabilidade dos serviços oferecidos pelo \textit{Firebase}, esse conjunto de tecnologias é uma escolha comum para aplicações móveis que utilizam \textit{Flutter}. A comunidade de desenvolvimento possui diversos tutoriais e documentações que abrangem os mais diversos casos de uso que podem ser utilizados em conjunto, o que facilitou o processo de desenvolvimento.

Cada um dos recursos oferecidos pela plataforma são explicados no Capítulo \ref{chp:infraestrutura}, e a forma de como foram utilizados está descrita no Capítulo \ref{chp:desenvolvimento}.

\chapter{Desenvolvimento}\label{chp:desenvolvimento}

Para realização da solução proposta nesse trabalho, é necessário o desenvolvimento de um aplicativo móvel para que o rastreamento de doenças infecciosas ocorra de forma automática. Assim, na Seção \ref{sec:requisitos} são apresentados os requistos funcionais e não funcionais do sistema, junto com a análise de cada um deles. A Seção \ref{sec:modelagem} tem o propósito de apresentar todo o processo de modelagem do sistema, discutindo sobre o racional por trás de cada modelagem e apresentando os diagramas construídos. Na Seção \ref{sec:uiux} são apresentadas as interfaces construídas para o aplicativo. A Seção \ref{sec:implementacao} contém a implementação das interfaces, modelagens e infraestrutura do sistema, utilizando o fluxo de trabalho \textit{GitFlow}.

\section{Levantamento e Análise de Requisitos}\label{sec:requisitos}
O aplicativo deve automatizar todo o ciclo do rastreamento, desde a captura da exposição à uma doença, até a notificação feita à pessoa possivelmente exposta. Com isso, a partir do problema de rastreamento de contatos apresentado no Capítulo \ref{chp:Introducao}, o processo de análise de requisitos foi feito pensando nos possíveis casos de uso do usuário.

Começando pelos requisitos não funcionais, a eficácia da solução depende diretamente do número de usuários que utilizarem o aplicativo. Por conta disso, o aplicativo deve funcionar em diferentes tipos de sistema operacionais de dispositivos móveis, a fim de atingir a maior quantidade de usuários.

Para que esse requisito seja atendido, o aplicativo pode ser desenvolvido especialmente para cada tipo de \Sigla{Sistema Operacional}{SO} ou utilizando um \textit{framework} híbrido, em que a partir da mesma base de código é possível construir os arquivos para cada SO.

Levando em consideração o tempo de desenvolvimento do aplicativo e o número de pessoas envolvidas, a utilização de um \textit{framework} híbrido é a melhor forma para atingir o maior número de usuários.

Além disso, como o aplicativo armazena os dados de localizações dos usuários, ele deve se preocupar em garantir a segurança dos mesmos. Essa segurança será garantida a partir da utilização de criptografia. Ela será aplicada aos dados em trânsito e em repouso, ou seja, tanto durante a comunicação entre cliente e servidor quanto no armazenamento no BD.

Ainda relacionado à segurança, a fim de preservar a privacidade das pessoas que utilizarem o aplicativo, a aplicação ou os usuários não devem ser capazes de descobrir quem possivelmente expôs outras pessoas a alguma doença. Isso significa que quando um usuário receber uma notificação de exposição, ele não será capaz de identificar quem foi, nem o local que isso ocorreu.

Para isso, o aplicativo não terá a inserção de nenhum dado pessoal, ou seja, a autenticação será feita de forma anônima, onde cada usuário será representado por uma cadeia de caracteres aleatórios.

Por último, toda localização terá prazo de validade de 14 dias. Isso significa que todo local visitado pelo usuário será armazenado apenas durante esse prazo. Como a solução do problema é a automação do rastreamento de contatos, localizações mais antigas do que esses dias não são necessárias, já que o período de transmissão do vírus haveria terminado.

A partir do que foi explicado nos parágrafos anteriores, foram levantados 3 requisitos não funcionais. Esses requisitos são:

\begin{enumerate}
  \item Compatibilidade multiplataforma;
  \item Garantia de privacidade e segurança dos dados;
  \item Exclusão de localizações antigas;
\end{enumerate}

Como o aplicativo usará um \textit{framework} multiplataforma para atender os requisitos não funcionais citados anteriormente, essa escolha trás um malefício: recursos nativos de cada dispositivo, como por exemplo o \textit{Bluetooth}, não possuem o mesmo suporte caso fosse uma codificação utilizando uma linguagem nativa.

Levando isso em consideração, os contatos que eventualmente ocorrerem entre diferentes usuários do aplicativo podem ser rastreados a partir de duas principais formas: 

\begin{itemize}
  \item Armazenamento dos locais visitados pelos usuários utilizando um sistema de \textit{check-in}, onde o usuário fará a inserção do local visitado utilizando um sistema de busca;
  \item Armazenamento dos contatos que foram efetuados utilizando o \textit{Bluetooth}. O dispositivo detectará os sinais transmitidos ao seu redor, e fará a persistencia desses contatos;
\end{itemize}

Ambas soluções solucionam o problema de formas distintas, porém, levando em consideração o requisito de compatibilidade multiplataforma e do baixo acesso à recursos nativos pelos \textit{frameworks}, a solução utilizada neste trabalho será a partir do sistema de \textit{check-in}.

Com isso, os usuários necessitam de um buscador para encontrar os locais que foram visitados. Esse buscador deve estar relacionado a um extenso BD para que a maioria dos locais inseridos pelos usuários sejam encontrados. Por conta disso, um BD externo deve ser utilizado, já que a criação de uma nova não traria a experiência esperada para o usuário.

Depois que a busca foi feita, o usuário deve ser capaz de salvar o local encontrado. O salvamento deve ser feito em um BD remoto, para que a rotina de checagem do rastreamento de contatos seja feita no servidor. Com isso, o dispositivo móvel do usuário não precisará de conectividade com a \textit{internet}, nem que o aplicativo rode em segundo plano durante a execução da rotina, evitando o consumo desnecessário de bateria do dispositivo.

Se um usuário for infectado por uma doença, ele deve ser capaz de declarar no aplicativo que está infectado. A partir disso, a rotina de rastreamento de contatos deve ser capaz de relacionar os locais do usuário infectado com os usuários comuns, procurando por algum possível contato que possa ter ocorrido.

O usuário deve receber notificações caso o mesmo tenha tido algum possível contato com outras pessoas que utilizam o aplicativo e estavam infectadas. Essa notificação deve ser enviada de forma automática e sem a necessidade de intervenção humana, para que o problema seja resolvido de forma automática de ponta a ponta.

A partir de toda análise acima, foram levantados 4 requisitos funcionais, que são:

\begin{enumerate}
  \item Buscar localizações;
  \item Salvar localizações;
  \item Declarar caso de infecção confirmado;
  \item Enviar notificação de exposição;
\end{enumerate}

\section{Modelagem}\label{sec:modelagem}
% https://proandroiddev.com/clean-architecture-data-flow-dependency-rule-615ffdd79e29
% https://github.com/android10/Android-CleanArchitecture/issues/136
% https://blog.cleancoder.com/uncle-bob/2012/08/13/the-clean-architecture.html
% https://resocoder.com/2019/08/27/flutter-tdd-clean-architecture-course-1-explanation-project-structure/

O processo de modelagem foi feito desde os aspectos mais abstratos do sistema, utilizando os diagramas de componentes e de pacotes, até os mais específicos, utilizando os diagrama de sequência e de atividades.

Para melhor apresentação e discussão de cada uma das modelagens, elas estão dividas nas subseções abaixo.

\subsection{Modelagem dos componentes do sistema}
% Diagrama de componentes
Esta Seção conterá a modelagem do diagrama de componentes, que será responsável por representar o funcionamento e a relação dos diferentes componentes do sistema, ou seja, a relação entre o aplicativo móvel utilizado pelo usuário, com o BD, \textit{backend} e o sistema de mensageria.

\subsection{Modelagem da arquitetura do \textit{software}}
% Modelagem mais abstrata da arquitetura: pacotes
% Modelagem mais específica: classes
Como explicado na Seção \ref{sec:clean}, o aplicativo desenvolvido neste trabalho utiliza os princípios da arquitetura limpa. Por conta disso, a base da arquitetura do \textit{software} é a \Figura{fig:clean1}.

Antes de começar a diagramação, é importante entender a diferença entre dependência e fluxo de dados do aplicativo, porque essa diferença é crucial para que a modelagem não quebre nenhuma das regras da arquitetura limpa.

A dependência do código acontece no tempo de compilação, ou seja, só há dependência entre os componentes se houver uma referência direta para outro componente.

Por exemplo, imagine que em uma das classes da camada \textit{Application Business Rules} do aplicativo possua um caso de uso que em sua implementação exista uma chamada para um método de uma classe presente na camada \textit{Interface Adapters}. Nesse exemplo, o caso de uso possuiria dependência com a camada \textit{Interface Adapters}, e estaria ferindo a regra de dependência, já que essa é uma camada externa em relação à \textit{Application Business Rules}.

Já o fluxo de dados seria equivalente ao fluxo de chamadas dentro do código, que não acontece no tempo de compilação, e sim no de execução. Pode parecer contraditório, mas a dependência nem sempre reflete o fluxo de dados do código.

Utilizando o mesmo exemplo anterior, é possível fazer com que o caso de uso não dependa da camada \textit{Interface Adapters}, mas ainda consiga atingir o mesmo comportamento esperado. Para que isso aconteça, uma interface deve ser criada na camada \textit{Application Business Rules}, e ela conterá as assinaturas dos métodos que o caso de uso necessita. Na camada \textit{Interface Adapters}, uma classe implementará os métodos definidos na interface através do relacionamento de herança.

Dessa maneira, através do \textit{Dependency Inversion Principle}, explicado na subseção \ref{sec:D}, a dependência existirá da camada \textit{Interface Adapters} para a camada \textit{Application Business Rules} por conta da herança, mas no fluxo de chamadas em tempo de execução, será da \textit{Application Business Rules} para a camada \textit{Interface Adapters}.

Com a diferença entre dependência e fluxo de dados compreendida, toda modelagem da arquitetura do sistema deve fazer com que nenhuma camada interna dependa de uma mais externa à ela. Para isso, todas as fronteiras das camadas haverão interfaces para possibilitar a comunicação sem que a regra de dependência seja ferida.

Como o aplicativo fará chamadas para componentes de persistência de dados, utilizando APIs para busca de locais e um BD remoto para o armazenamento, essas chamadas serão realizadas utilizado o padrão de repositório. A partir disso, o fluxo dos dados entre os componentes do sistema serão nessa ordem:

\begin{enumerate}
  \item \textit{View} faz chamadas dos métodos da \textit{View Model};
  \item \textit{View Model} executa o caso de uso;
  \item Caso de uso combina os dados dos repositórios das entidades;
  \item Cada repositório retorna os dados dos \textit{Data Sources};
  \item Os dados voltam para a \textit{View} e são mostrados ao usuário;
\end{enumerate}

A partir dessa ordem, a adaptação da \Figura{fig:clean1} com o padrão de repositório e da diferença entre dependência e fluxo de dados é feita na \Figura{fig:cleanadapt}.

\begin{figure}[!htb]
  \centering
  \includegraphics[scale=0.6]{clean-adapt.png}
  \caption{Diferença entre dependência e fluxo de dados.}
  \label{fig:cleanadapt}
\end{figure}

Note que as dependências estão sempre apontando para o interior. Levando em consideração a explicação de como as fronteiras das camadas são passadas através da inversão de dependências, o diagrama de pacotes da \Figura{fig:package} possui a modelagem mais abstrata e de maior alto nível da arquitetura do aplicativo. 

% imagem do diagrama de pacotes 
\begin{figure}[!htb]
  \centering
  \includegraphics[scale=0.7]{Diagrama de pacotes.png}
  \caption{Diagrama de pacotes do aplicativo.}
  \label{fig:package}
\end{figure}

Para que a associação do diagrama da \Figura{fig:package} com as camadas e regras de dependência da \Figura{fig:clean1} sejam melhor visualizados, a \Figura{fig:package2} representa as camadas da arquitetura limpa, com as mesmas colorações, por cima do diagrama de pacotes.

\begin{figure}[!htb]
  \centering
  \includegraphics[scale=0.7]{package-adapt.png}
  \caption{Diagrama de pacotes com as colorações das camadas da arquitetura limpa.}
  \label{fig:package2}
\end{figure}

\subsection{Modelagem dos casos de uso}
% Modelagem geral: casos de uso
% Modelagem de cada caso de uso: atividade e sequência
Os casos de uso do sistema são herdados do levantamento de requisitos. Por conta disso, a partir do que foi discutido na Seção \ref{sec:requisitos}, o diagrama de casos de uso está representado pela \Figura{fig:usecasesdiagram}.

\begin{figure}[!htb]
  \centering
  \includegraphics[scale=0.7]{Diagrama de casos de uso.png}
  \caption{Diagrama de casos de uso.}
  \label{fig:usecasesdiagram}
\end{figure}

Cada caso de uso da \Figura{fig:usecasesdiagram} é uma funcionalidade. Para cada uma delas serão modelados os diagramas de atividade e de sequência.

A modelagem do diagrama de atividades possuirá o objetivo de entender o processo e os comportamentos de cada funcionalidade. Já o diagrama de sequência descreverá como e em que ordem um grupo de objetos trocam mensagens entre si, ou seja, a forma que eles funcionam em conjunto.

\subsection{Modelagem de dados}
% https://www.youtube.com/watch?v=35RlydUf6xo
% https://medium.com/flutterdevs/firebase-data-modeling-tips-2fc61724743a
% https://fireship.io/courses/firestore-data-modeling/

Esta Seção conterá as informações e discussões sobre a modelagem de dados do BD \textit{Firestore}, com o objetivo de entender o motivo dos dados estarem dispostos da maneira que foram modelados.

\section{Interfaces de usuário do aplicativo}\label{sec:uiux}
Esta Seção conterá os \textit{layouts} das telas construídas no aplicativo móvel.

\section{Implementação}\label{sec:implementacao}
Esta Seção conterá as informações sobre como a interface de usuário, as modelagens e a infraestrutura do sistema foram construídas e configuradas, contendo exemplos de trechos da implementação.

Além disso, também mostrará como o projeto foi gerenciado no \textit{GitHub}, mostrando o quadro \textit{Kanban} utilizado, com a criação das \textit{issues} - que representam as tarefas que deveriam ser concluídas - e mostrando um exemplo de funcionamento prático do fluxo de trabalho \textit{GitFlow}.

\chapter{Infraestrutura}\label{chp:infraestrutura}

Para a realização da solução proposta neste trabalho, recursos de infraestrutura são necessários para que seja possível a execução de rotinas de rastreamento de contatos no servidor, utilização do BD e da comunicação entre o cliente e o servidor. Assim, na Seção \ref{sec:firebase} será explicado o que é e para que serve um serviço de \textit{backend}, quais recursos de infraestrutura serão utilizados e o que são os sistemas de autenticação e o envio de mensagens. Na Seção \ref{sec:firestore} será explicado sobre a solução de BD da plataforma \textit{Firebase} e o que cada entidade dele significa. Na Seção \ref{sec:cloudfunctions} será explicado o que é a \textit{Cloud Function} e qual será a responsabilidade dela neste projeto. Por fim, na Seção \ref{sec:pubsub} será explicado sobre a solução de mensageria da GCP, o \textit{Pub/Sub}, em que serão abordadas as entidades relacionadas a esse serviço, qual a responsabilidade de cada uma delas e qual será a responsabilidade desse serviço neste trabalho.

\section{Serviço de \textit{Backend}}\label{sec:firebase}

A infraestrutura da aplicação será totalmente em nuvem e os principais motivos dessa decisão foram: as faturas de cobrança estão diretamente relacionadas à carga de utilização, e como o escopo deste trabalho é um teste de conclusão de curso, essa carga será baixa; totalmente gerenciado por grandes empresas; boa documentação e facilidade na criação de ambientes.

A nuvem que será utilizada neste trabalho é a \Sigla{\textit{Google Cloud Platform}}{GCP} porque ela entrega serviços que atendem perfeitamente os requisitos do trabalho, por exemplo, APIs de busca de locais que utilizam o BD do \textit{Google}.

Neste projeto será utilizado um \Sigla{\textit{Backend-as-a-Service}}{BaaS}, que é um serviço de \textit{backend} gerenciado por uma empresa. Como a provedora de nuvem escolhida nesse projeto é a GCP, o serviço de BaaS dela se chama \textit{Firebase}.

\textit{Firebase} é uma plataforma digital que tem como objetivo acelerar o processo de desenvolvimento de aplicativos e oferecer serviços que atendam diversos requisitos, como a criação de BDs escaláveis, sistemas de autenticação, sistemas de envio de notificações, entre outros serviços.

O \textit{Firebase} está diretamente relacionado com a GCP. Todo projeto criado nele também está representado na GCP, que é a base dos serviços de computação em nuvem da \textit{Google}. A diferença é que o \textit{Firebase} abstrai grande parte da configuração da infraestrutura, gerenciando as configurações necessárias para que os serviços funcionem automaticamente, sem que o usuário precise entender sobre cada detalhe que acontece em segundo plano.

Alguns serviços existem nas duas plataformas, e no fundo são os mesmos, apenas com interfaces de consumo diferentes. Por exemplo, o serviço de \textit{Cloud Functions} pode ser utilizado em ambas plataformas, e quando criado no \textit{Firebase}, também é possível visualizá-lo na interface da GCP.

Porém, algumas configurações não são possíveis de serem feitas diretamente pelo \textit{Firebase}. Um exemplo disso é a gestão de identidades, que é responsável por definir quais permissões cada usuário tem dentro da plataforma e só pode ser configurada pela GCP.

A principal vantagem competitiva do \textit{Firebase} quando comparado a outras plataformas é a alta produtividade que ele traz ao desenvolvimento de aplicativos.

Os serviços do \textit{Firebase} que serão utilizados neste trabalho são o sistema de autenticação, o envio de notificações, o BD e as \textit{Cloud Functions}. Em conjunto a esses serviços, também será utilizado um recurso exclusivo da GCP, que é o \textit{Pub/Sub}.

Começando pela autenticação, ela é a peça chave para que os locais visitados pelos usuários sejam salvos em seus respectivos documentos no BD, dando uma experiência personalizada no aplicativo. Essa autenticação pode ser feita de diversas formas diferentes, algumas delas são o \textit{login} utilizando \textit{e-mail} e senha, número de telefone celular, redes sociais e autenticação anônima.

A autenticação anônima, que é o único método de autenticação utilizado pelo aplicativo, é criado pelo próprio \textit{Firebase}. Ela é necessária somente para que o aplicativo diferencie os usuários para efetuar o rastreamento, sem precisar de informações pessoais do mesmo.

Outro serviço que será utilizado é o \Sigla{\textit{Firebase Cloud Messaging}}{FCM}, que é responsável pelo envio de notificações aos usuários que forem possivelmente expostos à alguma doença infecciosa.

O FCM é bastante flexível e possui algumas funcionalidades que atendem diferentes casos de uso. A principal funcionalidade são as formas que a notificação pode ser enviada para o cliente, que são para dispositivos únicos, grupos de dispositivos ou para os dispositivos inscritos em tópicos. 

O envio de notificações deve ser feito de forma automática, para isso, os outros 3 recursos de infraestrutura trabalharão em conjunto para que a implementação da lógica do envio de mensagens seja feita com sucesso.

\section{\textit{Firestore}}\label{sec:firestore}

O \textcite{FirestoreDocs} é um BD não relacional orientada a documentos oferecido pela \textit{Google}. As principais características dele são:

\begin{itemize}
    \item \textit{Serverless}: sem servidor, totalmente gerenciado por tecnologias da Google e com escalonamento automático para atender qualquer carga de requisições;
    \item Mecanismo de consulta avançado: permite a execução de transações com \Sigla{Atomicidade, Consistência, Isolamento e Durabilidade}{ACID} nos dados dos documentos armazenados;
    \item Segurança: integração com a autenticação do \textit{Firebase} para possibilitar controles de acesso de segurança com base na identidade;
    \item Replicação multirregional: redundância de BD em diversos \textit{data centers} espalhados pelo mundo com alta consistência, aumentando a disponibilidade do serviço, mesmo em caso de desastres;
    \item Flexibilidade: o modelo de dados disponibiliza a criação de estruturas hierárquicas flexíveis, onde os dados são armazenados em documentos, e estes organizados em coleções;
    \item Integração: o \textit{Firestore} possui integração perfeita com outros produtos do \textit{Firebase} ou da GCP.
\end{itemize}

O modelo de dados do \textcite{FirestoreDataModel} segue o formato JSON e é estruturado em entidades denominadas documentos e coleções. A \Figura{fig:exemplofirestorejson} demonstra um exemplo dessa estrutura.

\begin{figure}[!htb]
    \centering
    \includegraphics[scale=0.3]{Estrutura Firebase JSON - Carbon Code (estilo Seti).png}
    \caption{Exemplo de estrutura JSON com anotações referentes ao \textit{Firestore}.}
    \label{fig:exemplofirestorejson}
\end{figure}

Documento é a unidade de armazenamento do \textit{Firestore}, com campos que são mapeados para valores. Esses valores podem ser de diversos tipos: números inteiros, booleanos, \textit{timestamps}, \textit{strings} e até estruturas de dados como listas e mapas.

Cada documento é identificado por um nome e não podem possuir outros documentos criados diretamente dentro dele.

Como o \textit{Firestore} não possui nenhum esquema, o desenvolvedor tem total liberdade sobre quais campos colocar em cada documento e quais tipos de dados esses campos armazenam.

Utilizando os mesmos valores e estrutura da \Figura{fig:exemplofirestorejson}, a \Figura{fig:explicacaodocumentofirestore} representa um exemplo de documento do \textit{Firestore}.

\begin{figure}[!htb]
    \centering
    \includegraphics[scale=0.6]{documento-firestore.png}
    \caption{Exemplo de documento do \textit{Firestore}.}
    \label{fig:explicacaodocumentofirestore}
\end{figure}

As coleções são recipientes que armazenam um conjunto de documentos e são referenciadas pelo seu nome, da mesma forma que os documentos.

Os documentos dentro das coleções devem ter identificadores únicos, ou seja, dois documentos diferentes não podem possuir o mesmo nome dentro da mesma coleção. Por conta disso, é comum que os documentos sejam nomeados com o \Sigla{Identificador}{ID} do objeto que ele representa.

É importante ressaltar que coleções não podem armazenar dados diretamente, apenas através de documentos. Seguindo o mesmo raciocínio, os documentos não podem armazenar outros documentos. Cada entidade possui responsabilidades específicas. Por conta disso, para criar uma estrutura hierárquica no \textit{Firestore}, as coleções armazenam exclusivamente documentos, e os documentos, além dos campos chaves-valor, podem armazenar subcoleções - que possuem as mesmas características de uma coleção.

Para melhor visualização da estrutura hierárquica do \textit{Firestore}, a estrutura da \Figura{fig:exemplofirestorejson} está representada na \Figura{fig:explicacaofirestorecompleto}.

\begin{figure}[!htb]
    \centering
    \includegraphics[scale=0.43]{Estrutura Firestore.png}
    \caption{Exemplo de estrutura hierárquica do \textit{Firestore}}
    \label{fig:explicacaofirestorecompleto}
\end{figure}

\section{\textit{Cloud Functions}}\label{sec:cloudfunctions}

A \textit{Cloud Function} é o produto de função como serviço para criar aplicações com base em eventos, disponibilizado pelas plataformas \textit{Firebase} e GCP. Ela é uma maneira de ampliar o comportamento de um aplicativo e integrar com outros recursos disponíveis na plataforma por meio da adição de código no servidor. Com base nisso, ela serve como camada conectiva, disponibilizando a construção de lógicas entre os serviços da plataforma por meio da detecção e da resposta a eventos.

No caso de uso deste trabalho, a \textit{Cloud Function} consome os eventos publicados no \textit{Pub/Sub}, tratando-os através da execução do trecho de código implementado na função.

\section{\textit{Pub/Sub}}\label{sec:pubsub}

\textcite{PubSubDocs} é um serviço de mensagens assíncronas, que oferece um armazenamento de mensagens com alta disponibilidade, desempenho e grande escala. Esse serviço funciona através do conjunto dos 4 conceitos abaixo:

\begin{itemize}
    \item Mensagem: combinação de dados e atributos enviados a um tópico pelo \textit{publisher};
    \item Atributo de mensagem: par chave-valor que o \textit{publisher} pode definir para uma mensagem;
    \item Tópico: recurso para o qual os \textit{publishers} enviam as mensagens;
    \item Assinatura: recurso que representa o fluxo de mensagens que deve ser entregue aos \textit{subscribers} de um tópico específico;
\end{itemize}

No caso deste trabalho, o \textit{Pub/Sub} foi utilizado para integrar os eventos de escrita do \textit{Firestore} com a rotina de rastreamento de contatos, que foi implementada na \textit{Cloud Function}.


\chapter{Conclusões}\label{chp:conclusoes}

Este Capítulo conterá as discussões finais referentes ao problema e solução propostas neste trabalho.

% Comandos para incluir as referências bibliográficas
% Define espaçamento simples em cada referência
\begin{singlespacing}

% Adiciona uma linha em branco entre duas referências
\setlength\bibitemsep{10pt}   
%
% Adiciona as referências bibliográficas.
% Mude o título (title), caso o texto seja em inglês 
% ou espanhol.
\printbibliography[heading=bibintoc, % Adiciona no sumário
                   title={Referências bibliográficas} % Nome do Capítulo
                  ]
\end{singlespacing}

% Os anexos, se houver, vêm depois das referências:
%\appendix

% O comando a seguir inclui o arquivo apendices.tex
% que contém os apêndices. Observe o comando \appendix
% na linha anterior
% Detalhe: não precisa incluir a extensão .tex
%\include{apendices}
%
\end{document}
\chapter{Trabalhos correlatos}\label{chp:correlatos}

Este capítulo trata das pesquisas realizadas procurando soluções no mercado que se propõem a resolver o problema apresentado pelo projeto e também consultas por algumas palavras-chave no BD do \Sigla{Instituto Nacional da Propriedade Industrial}{INPI}.

\section{Busca por patentes de software registradas}\label{sec:inpi}
A consulta no INPI foi feita utilizando a categoria de Programas de Computador, buscando os que tinham como parte de seu título as palavras-chave rastreamento de doenças, exposição a doenças, aplicativo de rastreamento ou rastreamento de contatos.

Levando em consideração a ordem de busca apresentada, apenas a terceira encontrou algum resultado, totalizando três respostas. Uma delas é um aplicativo de celular e as outras duas são aplicativos de análise de mercado da plataforma \textit{NinjaTrader8}.

Analisando esse aplicativo móvel, ele foi registrado por \textcite{moope} no dia 27 de outubro de 2020 e se chama “MOOPE SISTEMA DE RASTREAMENTO E GESTÃO DE FROTAS - Módulo Aplicativo ios/Android, Módulo Frotas, Modulo Central Comandos, Modulo Central Notificações, Modulo MeusMoopes”, que tem como campo de aplicação a gestão de frotas. Portanto, o escopo deste aplicativo difere do proposto por este trabalho.

\section{Busca nas lojas de aplicativos}\label{sec:mercado}
No começo da pandemia do coronavírus, as empresas Google e Apple, detentoras dos dois principais sistemas operacionais para celulares - o \textit{iOS} e \textit{Android} - divulgaram o desenvolvimento de uma \Sigla{\textit{Application Programming Interface}}{API} para monitorar a exposição das pessoas ao coronavírus \cite{GoogleApple2020}. Esse monitoramento é feito a partir da captação do sinal entre pessoas próximas utilizando a tecnologia \textit{Bluetooth}. Essa API identifica se pessoas permaneceram próximas durante alguns minutos e se o sinal entre elas for suficientemente forte, cada dispositivo salva o identificador do outro que tiveram contato, assim seria possível fazer o rastreamento de contatos.

A API só pode ser consumida pelo governo dos países que se inscrevem no programa, ou seja, apenas os aplicativos desenvolvidos pelas autoridades públicas podem utilizar a tecnologia criada pelas duas empresas. Além disso, as empresas divulgaram que o recurso será desativado após a pandemia de coronavírus.

O único aplicativo que funciona no Brasil que possui a finalidade de rastreamento de contatos e que utiliza a API citada anteriomente foi desenvolvido pelo DATASUS e se chama "Coronavirus - SUS"  \cite{CoronavirusSUS}.

As principais funcionalidades deste aplicativo são que os usuários podem consultar as notícias oficiais compartilhadas pelo Ministério da Saúde, conferir as principais notícias falsas em circulação, informações sobre a doença, o compartilhamento de teste positivo e o rastreamento de contatos.

Além deste aplicativo, foram encontrados alguns desenvolvidos por outros países: o \textit{CDC}, \textit{Care19}, \textit{Crush Covid RI} e \textit{FEMA}. Nenhum desses aplicativos funcionam corretamente em território brasileiro, a única funcionalidade que pode ser aproveitada são as informações sobre a doença, escritas em inglês.

Com isso, o aplicativo que possui o objetivo de rastreamento de contatos e que funciona corretamente no Brasil é o Coronavírus - SUS.

As principais diferenças entre o aplicativo Coronavírus - SUS e o desenvolvido neste trabalho são: 
\begin{itemize}
    \item O aplicativo deste trabalho pode ser utilizado para qualquer tipo de doença infecciosa, enquanto o outro possui o foco no coronavírus;
    \item A API que o aplicativo Coronavírus - SUS utiliza será desativada após a pandemia, já a solução deste trabalho não possui nenhum componente temporário, podendo ser utilizado mesmo após a pandemia;
    \item O método de rastreamento do aplicativo deste trabalho utiliza o histórico de locais visitados e busca encontros entre os usuários, já o do SUS utiliza o sinal \textit{Bluetooth} para detectar o contato entre os usuários;
\end{itemize}

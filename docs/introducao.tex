\chapter{Introdução}\label{chp:Introducao}
% contexto do trabalho, motivações, objetivos e estrutura dos capítulos

A pandemia de COVID-19 mudou radicalmente as percepções do mundo em relação a propagação de doenças infecciosas como uma importante ameaça à humanidade. De acordo com o Instituto Ipsos \cite{Gebrekal2021}, o coronavírus se estabeleceu como a maior preocupação da população mundial por mais de um ano consecutivo.

Durante uma conferência realizada em Vancouver, \textcite{Gates2015} afirmou que especialistas alertavam que o mundo não estava preparado para o surgimento de um novo vírus com alta taxa de contágio. A facilidade que a humanidade possui em se locomover rapidamente para qualquer lugar do mundo através de aeroportos, ferrovias e estradas, a falta de um sistema de combate ao proliferamento de doenças infecciosas, métodos de prevenção e equipes de epidemiologistas altamente treinados contribuíram para o crescimento descontrolado dos casos de COVID-19 no ano de 2020.

Com o objetivo de diminuir a propagação do vírus, as mídias sociais foram bombardeadas com informações sobre métodos de prevenção, tais como lavar as mãos, evitar o compartilhamento de objetos pessoais, uso de máscaras, isolamento social e o rastreamento de contatos.

O rastreamento de contatos tem o objetivo de encontrar as pessoas que tiveram contato com as que foram infectadas pelo vírus, e instruí-las para não se encontrarem com família e amigos, ir aos mercados, bancos, ou seja, permanecerem isolamento social em suas casas, tudo isso com o objetivo de evitar que a corrente de transmissão do vírus continue e diminuir a sua taxa de contágio.

\section{Problema}\label{sec:problema}
Apesar do rastreamento reduzir a propagação do vírus avisando as pessoas expostas que elas podem estar infectadas, existia um grande problema: tudo é feito manualmente. Isso envolve um custo humano muito alto, são necessários muitos profissionais para que o rastreamento tenha algum resultado expressivo.

Além do alto custo para o governo manter a operação funcionando, ela depende integralmente da memória do paciente. Se o paciente entrevistado não lembrar de todas as pessoas que teve contato, ou se ele não souber os dados referentes às pessoas que ele esteve nas últimas semanas, isso afeta diretamente no resultado desse método de prevenção. Se muitas pessoas que foram expostas não forem identificadas, a corrente de transmissão continuará crescendo.

\section{Solução proposta}\label{sec:solucao}
Quando o número de pessoas envolvidas em uma solução é alto demais para arcar com os custos ou quando existe um alto risco humano envolvido, que nesse caso é o risco relacionado à memória do paciente, uma tecnologia de automação surge como solução, e nesse caso não é diferente.

A solução proposta é um aplicativo celular que será responsável tanto pela coleta das localizações visitadas quanto pela notificação de possíveis exposições aos usuários.

O aplicativo funcionará da seguinte maneira: o usuário deverá fazer \textit{check-in} nos locais que visitar e o aplicativo será responsável pelo armazenamento de cada local e o horário correspondente da visita. Caso algum usuário for infectado por alguma doença contagiosa, ele informará que está infectado e a automação fará uma busca no banco de dados. Caso outro usuário tenha visitado o mesmo local durante o mesmo intervalo de tempo, ele receberá uma notificação informando-o que pode ter sido exposto à uma doença infecciosa e sugerindo que evite sair de sua casa.

\section{Estrutura do texto}\label{sec:estrutura}

O projeto foi dividido em 7 capítulos e cada um deles possui uma responsabilidade específica.

O Capítulo \ref{chp:metodologia} explica quais foram os métodos utilizados para o desenvolvimento deste trabalho. O Capítulo \ref{chp:correlatos} contém a busca por patentes registradas e de aplicações correlatas ao tema deste projeto. No Capítulo \ref{chp:referencial} é feita a apresentação das tecnologias utilizadas e das pesquisas feitas sobre o tema. O Capítulo \ref{chp:desenvolvimento} contém o levantamento e análise de requisitos, a modelagem dos diagramas e da arquitetura utilizada no desenvolvimento e a explicação da implementação das funcionalidades. O Capítulo \ref{chp:infraestrutura} explica quais recursos de infraestrutura são necessários para que a aplicação funcione e qual a relação entre cada um deles. Por fim, o Capítulo \ref{chp:conclusoes} contém as conclusões do trabalho.



% Você pode usar o site http://www.tablesgenerator.com
% para gerar as tabelas em LaTeX.


% Note que podemos incluir automaticamente alguns termos na lista de abreviaturas e siglas no pré-texto. Veja exemplos a seguir no código fonte. 

% A sigla \Sigla{abêcê}{ABC} aparecerá automaticamente na lista de abreviaturas e siglas no pré-texto. Do mesmo modo, podemos usar o comando \SiglaHifen{xisipsilonzê}{XYZ} separado por hífen ao invés de aparecer entre parênteses.


% Aqui começa uma Seção.
% Use o comando \label para definir um rótulo, 
% caso seja necessário referenciar essa seção
% posteriormente.
%\section{Exemplo de seção}\label{sec:exemplo_secao} 

%Agora observe como se faz uma citação de artigo científico em periódico \cite{Gradvohl2014c}. De acordo com \textcite{Gradvohl2016}, essa é uma citação direta. Se for citar mais de um trabalho, faça da seguinte forma \cite{Caldana2017,Gradvohl2015}. As referências bibliográficas estão no arquivo \texttt{bibliografia.bib}. Outros exemplos de citações também se encontram nesse arquivo.

% Veja a seguir o comando para criar uma figura e o resultado, na \Figura{fig:xwing}. Note, no código fonte, que no comando \texttt{caption} podemos estabelecer uma \enquote{legenda curta} para aparecer na Lista de Figuras. A legenda curta é opcional.

%\begin{figure}[!htb]
%\centering
%As figuras estão na pasta figuras.
% Se seu texto estiver demorando muito para compilar, 
% use figuras no formato PDF ou PNG.
%\includegraphics[scale=0.2]{starwars21280.jpg}
%\caption[Legenda curta de figura]{Legenda mais extensa de figura.}
%\label{fig:xwing}
%\end{figure}

%\subsection{Exemplo de subseção}
%É importante evitar chegar a esse nível de subseção. Dois níveis são suficientes. Use essa opção em último caso, apenas.


%\subsection{Exemplo de adição de siglas}\label{subsec:siglas}
%Para adicionar uma sigla ou abreviatura na lista de siglas e abreviaturas, use o comando ``\texttt{\textbackslash{}Sigla\{nome por extenso\}\{abreviatura\}}'' ou ``\texttt{\textbackslash{}SiglaHifen\{nome por extenso\}\{abreviatura\}}'' para adicionar a sigla com hífen. 
%Por exemplo, respectivamente, \Sigla{Ácido Desoxirribonucleico}{DNA} ou \SiglaHifen{Ácido Ribonucleico}{RNA}. A lista de siglas é adicionada automaticamente.

%\section{Comandos opcionais para facilitar}
%Este modelo também criou alguns comandos adicionais não apenas para facilitar o trabalho de quem escreve, mas também para manter uma formatação mais consistente.

%Entre esses comandos estão o \texttt{\textbackslash{}ie} que inclui a abreviatura ``\ie'' no texto (equivalente ao ``isto é''). Usar esse comando vai garantir que a abreviatura não se separe entre linhas e que o espaço entre o `.' e a próxima letra seja fixo. O mesmo vale para os comandos \texttt{\textbackslash{}eg} que inclui a abreviatura ``\eg'' e \texttt{\textbackslash{}pex} que inclui a abreviatura ``\pex''.

%Também existem os comandos \texttt{\textbackslash{}Capitulo\{rótulo do capítulo\}}, \texttt{\textbackslash{}Equacao\{rótulo da equação\}}, \texttt{\textbackslash{}Figura\{rótulo do figura\}}, \texttt{\textbackslash{}Secao\{rótulo da seção\}} e \texttt{\textbackslash{}Tabela\{rótulo da tabela\}}. Esses comandos inserem referências para os respectivos elementos. Além disso, no próprio texto aparece a \textit{string} (``Capítulo'', ``Equação'', ``Figura'' etc) seguida da referência já com o link. Por exemplo, \Secao{sec:exemplo_secao}. Sugere-se a utilização desses comandos para referenciar os respectivos elementos ao invés do comando \texttt{\textbackslash{}ref\{rótulo\}}. Assim, o texto ficará mais uniforme. 

%É possível também usar esses comandos nas versões no plurall para conjuntos de referências. Por exemplo, para referenciar várias seções, você pode utilizar o comando \texttt{\textbackslash{}secoes\{rótulo\_1, rótulo\_2, rótulo\_3\}}.

%Por exemplo, suponha que queiramos referenciar as \secoes{sec:exemplostabelas,sec:exemplo_secao,subsec:siglas}.